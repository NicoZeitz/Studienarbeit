\renewcommand{\abstractname}{Abstract} % Veränderter Name für das Abstract
\begin{abstract}
    \begin{addmargin}[1.5cm]{1.5cm}        % Erhöhte Ränder, für Abstract Look
        \thispagestyle{plain}                  % Seitenzahl auf der Abstract Seite

        \begin{center}
            \small\textit{- Deutsch -}             % Angabe der Sprache für das Abstract
        \end{center}

        \vspace{0.25cm}

        Diese Studienarbeit befasst sich mit Entwicklung einer leistungsfähigen Computerspielengine für das Brettspiel Patchwork, welche auf Basis eines Algorithmus aus maschinellem Lernen funktioniert, um das für zwei Personen ausgelegte Spiel dem einzelnen Spieler zu eröffnen und eine strategisch anspruchsvolle Erfahrung zu bieten. Die Arbeit umfasst eine spieltheoretische Analyse des Brettspiels und die prototypische Implementierung als Computerspiel. Als Grundlage für die Umsetzung der unterschiedlichen komplexeren Computerspielgegner werden die Algorithmen \acl{PVS}, \acl{MCTS} und AlphaZero verwendet. Anschließend werden die Computergegner für die Evaluierung untereinander und mit einfacheren Vergleichsspielern, einem Gegner mit zufälligen Zügen und ein Gegner basierend auf einem Greedy-Algorithmus, verglichen. Als Interaktionsmöglichkeit mit dem Computerspiel besteht eine Implementierung einer einfachen Benutzeroberfläche in der Kommandozeile, das Konzept und Design für eine intuitive, interaktive Benutzeroberfläche wird beschrieben.
        
    \end{addmargin}
\end{abstract}