\chapter{Fazit}
\label{chapter:fazit}

Im Rahmen der vorliegenden Studienarbeit konnte das Brettspiel Patchwork spieltheoretisch analysiert werden und darauf aufbauend das Spiel als Computerspiel erfolgreich realisiert werden. Das Brettspiel besitzt sehr viele mögliche Spielzüge und Spielzustände, wodurch eine optimale beziehungsweise sehr schnelle Implementierung der Computergegner im Computerspiel schwer zu erreichen ist. Fast alle Computerspielengines funktionieren entsprechend der Erwartungen an dieselben, allerdings existieren einige Einschränkungen. Der \ac{PVS} Spieler schneidet deutlich schlechter ab als der \ac{MCTS} Spieler und dieser Vorsprung ist durch die Verbesserungen, welche an der Spielengine gemacht werden können, schwer aufzuholen da bei der \ac{MCTS} Spielengine im gleichen Maße noch Optimierungen stattfinden können. Die Ausnahmen macht der PatchZero Spieler, da er die Erwartungen verfehlt. Bei der Spielengine fehlt vermutlich noch weiteres und längeres Training des neuronalen Netzes, um durch kontinuierliche Evaluation bei besseren Trainingsständen den Spieler besser einzuschätzen. In der Zukunft könnten weitere Computergegner-Typen, Variationen oder Verbesserungen der bestehenden Computergegner umgesetzt werden und auf Tauglichkeit für das Computerspiel getestet werden. Außerdem sollten für eine bessere Evaluation aller Computerspielengines die Anzahl der Spiele untereinander signifikant erhöht werden, um eine statistische Sicherheit bei der Bewertung der Spieler zu erhalten.

Das Ziel der Umsetzung des Computerspiels als interaktives System mit intuitiver Benutzeroberfläche wurde nicht erreicht. Zwar wurde das Brettspiel auf ein digitales Medium übertragen und das Spiel lässt sich als einzelner Spieler spielen, allerdings ist zum jetzigen Zeitpunkt nur eine Interaktion mit der \ac{CLI} Schnittstelle möglich, da die Benutzeroberfläche noch nicht finalisiert wurde. Dementsprechend kann nicht bewertet werden, inwieweit die nahtlose Transformation des Brettspiels zum Computerspiel gelungen ist.

Als weiteren Ausblick für die Zukunft der Studienarbeit könnte die grundlegende Umsetzung von Patchwork als Computerspiel weiterreichend nach Optimierungsmöglichkeiten gesucht werden, um durch Optimierung die Performance des Computerspiels zu verbessern. Außerdem könnte das neu erschlossene digitalen Medium des Computerspiels dazu verwendet werden, die Unabhängigkeit der Entfernung auszunutzen, um das ehemalige Brettspiel in ein kompetitives Online-Mehrspieler-Spiel umzuwandeln.