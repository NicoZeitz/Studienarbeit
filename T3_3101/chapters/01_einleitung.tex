\chapter{Einleitung}
\label{chapter:einleitung}

Die fortschreitende Entwicklung von Computern, intelligenten Geräten sowie von künstlicher Intelligenz (\acs{KI}) hat nicht nur den Alltag vieler Menschen nachhaltig verändert, sondern auch im Bereich der traditionellen Brettspiele wie Schach und Go signifikante Spuren hinterlassen. Der Ursprung dieser technologischen Revolution lässt sich auf die Pionierarbeit in den frühen Phasen der Computerära zurückverfolgen, als einfache heuristische Ansätze die ersten \acs{KI}-Gegner in strategischen Spielen hervorbrachten. Die chronologische Reise durch die Geschichte der \acs{KI}-Evolution in Spielen spiegelt nicht nur die wachsende Intelligenz von \acs{KI}-Gegnern wider, sondern auch die parallelen Fortschritte in der Hardware, die diese Entwicklungen ermöglichten.

Schon in den 1950er Jahren wurde ein Meilenstein gesetzt, indem die \enquote{\acs{IBM} 701 Electronic Data Processing Machine}, welches der erste kommerzielle wissenschaftliche Großrechner von \ac{IBM} war \cite{2023.IBM701}, erstmals für Schachprogrammierung verwendet wurde und ein Schachprogramm besaß. \cite{2014.EndeDerBescheidenheit}

Doch es sollte noch bis zum 11. Mai 1997 dauern, als der Schachweltmeister Garry Kasparov von \ac{IBM}s Schachcomputer Deep Blue in einem Rematch besiegt wurde. Bereits ein Jahr zuvor waren die beiden Gegner aufeinandergetroffen, damals ging aber noch der menschliche Spieler und russische Schachgroßmeister als Sieger aus dem Spiel hervor. Zu den Folgen dieses Ergebnisses zählt die Erkenntnis, dass Computer in der Lage sind, den Menschen mit seinem menschlichen Intellekt zu übertreffen. \cite{2022.DeepBlue}

Diese wegweisende Begegnung markierte außerdem den Übergang von heuristischen Ansätzen zu komplexen algorithmischen Methoden in der \ac{KI}-Entwicklung. In der Welt des Go-Spiels, das als besonders komplex gilt, dauerte es bis 2016, als AlphaGo, eine von der Google-Tochter Deepmind entwickelte \ac{KI}, einen der weltweit besten Go-Spieler, Lee Sedol, besiegte. Hier zeigte sich erneut ein deutlicher Fortschritt in den Fähigkeiten von Computern und \ac{KI}, da dieser Computergegner durch ein lernendes, künstliches neuronales Netz betrieben wurde. Diese Netze sind in der Lage selbstständig auf Problemlösungen und Strategien zu kommen, auf die Menschen zuvor nicht gekommen sind, denn die Spielweise von AlphaGo verblüffte viele Profi-Spieler und Go-Fans, da die Software Züge machte, die kein Mensch zuvor gespielt hatte. \cite{2016.AlphaGo}

\vspace*{-5cm}
\pagebreak

Die ständig wachsende Leistungsfähigkeit moderner Computer ermöglicht es, komplexe Entscheidungsprozesse und Strategien in Echtzeit zu berechnen, was wiederum zu immer anspruchsvolleren und adaptiven \ac{KI}-Gegnern führt.

\section{Motivation}
\label{chapter:motivation}

Die vorliegende Arbeit ist durch ein tiefes Interesse an der Interaktion zwischen Menschen und Computergegnern motiviert, wie sie in Spielen wie Schach aufgefunden werden kann. Nachzuvollziehen, wie eine \ac{KI} in der Lage ist, menschliche Spieler herauszufordern und zu besiegen, ist ein Antrieb dieser Arbeit. Dabei steht nicht nur die reine Leistungsfähigkeit der \ac{KI} im Vordergrund, sondern auch die Frage, wie eine anspruchsvolle Spielerfahrung für den menschlichen Gegner geschaffen werden kann.

Die Umsetzung des Brettspiels Patchwork als Computerspiel eröffnet neue Möglichkeiten und Herausforderungen. Es ermöglicht nicht nur das Spiel gegen einen künstlichen Gegner, sondern auch die Untersuchung, wie sich ein Brettspiel bestmöglich von einem analogen Medium auf ein digitales Medium übertragen lässt. Dieser weitere Antrieb der Arbeit ist die Schaffung eines interaktiven Systems, das eine nahtlose Transformation des Brettspiels in ein Computerspiel ermöglichen soll. Es soll hierbei besonders auf die Übertragung des Spielgefühls und der bereits bekannten Funktionsweise des Spiels geachtet werden, sodass bereits erfahrene Patchwork-Brettspieler ihre Fähigkeiten einwandfrei anwenden können. Außerdem wird durch die Umsetzung dieser Arbeit ermöglicht, dass ein Brettspiel, das ursprünglich zwei Spieler fordert, für einen einzigen Spieler zugänglich gemacht werden kann.

Zusammenfassend lässt sich sagen, dass diese Arbeit von der Neugier angetrieben wird, wie traditionelle Brettspiele durch Computergegner ergänzt werden können, um dem menschlichen Spieler eine weitere Herausforderung zu bieten. Diese Arbeit soll neben dem zuvor Erwähnten einen Blick hinter die Computer- und \ac{KI}-Gegner werfen, Funktionsweisen erläutern und ein tiefes Verständnis für diese zu schaffen.

\pagebreak

\section{Ziel und Vorgehen}
\label{chapter:ziel-und-vorgehen}

Das Ziel dieser Studienarbeit ist somit die Umsetzung des Brettspiels Patchwork als Computerspiel, wobei der Fokus vor allem auf der Erschaffung eines kompetenten Computergegners liegt. Hierzu werden unterschiedliche Herangehensweisen für Computergegner, welche auch als \ac{KI}-Gegner oder Computerspielengine bezeichnet werden können, implementiert und evaluiert.

Zuerst muss dafür jedoch das Brettspiel vollständig analysiert werden und in ein Computerprogramm umgewandelt und implementiert werden. Danach können die vier folgenden unterschiedlichen Ansätze der Computerspielengines erstellt werden. Der erste und voraussichtlich schwächste Gegner wird durch zufällige Spielzüge Patchwork spielen (Ansatz A). Mit Hilfe des Minimax-Algorithmus wird die Spielengine hinter Ansatz B den menschlichen Spieler herausfordern, während die Spielengine von Ansatz C den Algorithmus des Monte Carlo Tree Search verwenden wird. Der letzte Ansatz D wird abschließend das modernste Konzept verfolgen und ein selbstlernendes, neuronales Netz namens AlphaZero verwenden, um die besten Spielzüge zu berechnen.

Ein weiteres Ziel dieser Arbeit ist das Brettspiel Patchwork vom Tisch, möglichst intuitiv als Computerspiel als interaktives System umzusetzen. Hierbei liegt der Hauptfokus auf der Erstellung einer bestmöglichen interaktiven Benutzeroberfläche, bei der der erfahrene Spieler seine Kenntnisse und Gewohnheiten vom Brettspiel direkt am Computer anwenden kann.

Damit diese Arbeit von jedem interessierten Leser verstanden werden kann, folgt in den nächsten Kapiteln die Erläuterung aller nötigen Grundlagen dieser Arbeit. Abgeschlossen wird diese Arbeit mit der Evaluierung der unterschiedlichen Computerspielengines und der Betrachtung, ob das Brettspiel erfolgreich als interaktives System für den Computer umgesetzt werden konnte und ein Fazit der gesamten Studienarbeit.
