\chapter{Einleitung}
\label{chapter:einleitung}

Die fortschreitende Entwicklung von Computern, intelligenten Geräten sowie von \ac{KI} hat nicht nur den Alltag vieler Menschen nachhaltig verändert, sondern auch im Bereich der traditionellen Brettspiele wie Schach und Go signifikante Spuren hinterlassen. Der Ursprung dieser technologischen Revolution lässt sich auf die Pionierarbeit in den frühen Phasen der Computerära zurückverfolgen, als einfache heuristische Ansätze die ersten \ac{KI}-Gegner in strategischen Spielen hervorbrachten. Die chronologische Reise durch die Geschichte der \ac{KI}-Evolution in Spielen spiegelt nicht nur die wachsende Intelligenz von \ac{KI}-Gegnern wider, sondern auch die parallelen Fortschritte in der Hardware, die diese Entwicklungen ermöglichten.

Schon in den 1950er Jahren wurde ein Meilenstein gesetzt, indem die \enquote{\acs{IBM} 701 Electronic Data Processing Machine}, welcher der erste kommerzielle wissenschaftliche Großrechner von \ac{IBM} war, erstmals für Schachprogrammierung verwendet wurde und ein Schachprogramm besaß.

Doch es sollte noch bis zum 11. Mai 1997 dauern, als der Schachweltmeister Garry Kasparov von \ac{IBM}s Schachcomputer Deep Blue in einem Rematch besiegt wurde. Bereits ein Jahr zuvor waren die beiden Gegner aufeinandergetroffen, damals ging aber noch der menschliche Spieler und russische Schachgroßmeister als Sieger aus dem Spiel hervor. Zu den Folgen dieses Ergebnisses zählt die Erkenntnis, dass Computer in der Lage sind, den Menschen mit seinem menschlichen Intellekt zu übertreffen.

Diese wegweisende Begegnung markierte außerdem den Übergang von heuristischen Ansätzen zu komplexen algorithmischen Methoden in der \ac{KI}-Entwicklung. In der Welt des Go-Spiels, das als besonders komplex gilt, dauerte es bis 2016, als AlphaGo, eine von der Google-Tochter Deepmind entwickelte \ac{KI}, einen der weltweit besten Go-Spieler Lee Sedol besiegte. Hier zeigte sich erneut ein deutlicher Fortschritt in den Fähigkeiten von Computern und \ac{KI}, da dieser Computergegner durch ein lernendes, künstliches neuronales Netz betrieben wurde. Diese Netze sind offenbar in der Lage, selbstständig auf Problemlösungen zu kommen, auf die Menschen nicht gekommen sind, da die Spielweise von AlphaGo viele Profi-Spieler und Go-Fans verblüffte, auf Grund der Tatsache, dass die Software Züge machte, die kein Mensch machen würde.

Die ständig wachsende Leistungsfähigkeit moderner Computer ermöglicht es, komplexe Entscheidungsprozesse und Strategien in Echtzeit zu berechnen, was wiederum zu immer anspruchsvolleren und adaptiven \ac{KI}-Gegnern führt.

\section{Motivation}
\label{chapter:motivation}

TODO:

\section{Ziel und Vorgehen}
\label{chapter:ziel-und-vorgehen}

Das Ziel dieser Studienarbeit ist somit die Umsetzung des Brettspiels Patchwork als Computerspiel, wobei der Fokus vor allem auf der Erschaffung eines kompetenten Computergegners liegt. Hierzu werden unterschiedlich Herangehensweisen für Computergegner, welche auch als \ac{KI}-Gegner oder Computerspielengine bezeichnet werden können, implementiert und evaluiert.

Zuerst muss dafür jedoch erst das Brettspiel vollständig analysiert werden und das Brettspiel in ein Computerprogramm umgewandelt und implementiert werden. Danach können die vier folgenden unterschiedlichen Ansätze der Computerspielengines erstellt werden. Der erste und voraussichtlich schwächste Gegner und wird durch zufällige Spielzüge Patchwork spielen (Ansatz A). Mit Hilfe des Minimax-Algorithmus wird die Spielengine hinter Ansatz B den menschlichen Spieler herausfordern, während die Spielengine von Ansatz C den Algorithmus des Monte Carlo Tree Search verwenden wird. Der letzte Ansatz D wird abschließend das modernste Konzept verfolgen und ein selbstlernendes, neuronales Netz namens AlphaZero verwenden, um die besten Spielzüge berechnen.

Außerdem ist ein weiteres Ziel dieser Arbeit das Brettspiel Patchwork vom Tisch, möglichst intuitiv als Computerspiel als interaktives System umzusetzen. Hierbei liegt der Hauptfokus auf der Erstellung einer bestmöglichen interaktiven Benutzeroberfläche, bei welcher der erfahrene Spieler seine Kenntnisse und Gewohnheiten vom Brettspiel direkt am Computer anwenden kann.

Damit dieser Arbeit von jedem interessierten Leser verstanden werden kann, folgt in den nächsten Kapiteln die Erläuterung aller nötigen Grundlagen dieser Arbeit. Abgeschlossen wird dieser Arbeit mit der Evaluierung der unterschiedlichen Computerspielengines und der Betrachtung, ob das Brettspiel erfolgreich als interaktives System für den Computer umgesetzt werden konnte und ein Fazit der gesamten Studienarbeit.
